%\VignetteIndexEntry{flexsurv user guide}

%% TODO 
%% Search JSS and other journals for survival software.  JSS has been going for very long
%% survival task view
%% What examples of custom dist? llogis? tobit example in help(survreg)?  Walter's model
%% spline with haz as fn of time
%% fractional polynomials


\documentclass[nojss,nofooter]{jss}\usepackage[]{graphicx}\usepackage[]{color}
%% maxwidth is the original width if it is less than linewidth
%% otherwise use linewidth (to make sure the graphics do not exceed the margin)
\makeatletter
\def\maxwidth{ %
  \ifdim\Gin@nat@width>\linewidth
    \linewidth
  \else
    \Gin@nat@width
  \fi
}
\makeatother

\definecolor{fgcolor}{rgb}{0.345, 0.345, 0.345}
\newcommand{\hlnum}[1]{\textcolor[rgb]{0.686,0.059,0.569}{#1}}%
\newcommand{\hlstr}[1]{\textcolor[rgb]{0.192,0.494,0.8}{#1}}%
\newcommand{\hlcom}[1]{\textcolor[rgb]{0.678,0.584,0.686}{\textit{#1}}}%
\newcommand{\hlopt}[1]{\textcolor[rgb]{0,0,0}{#1}}%
\newcommand{\hlstd}[1]{\textcolor[rgb]{0.345,0.345,0.345}{#1}}%
\newcommand{\hlkwa}[1]{\textcolor[rgb]{0.161,0.373,0.58}{\textbf{#1}}}%
\newcommand{\hlkwb}[1]{\textcolor[rgb]{0.69,0.353,0.396}{#1}}%
\newcommand{\hlkwc}[1]{\textcolor[rgb]{0.333,0.667,0.333}{#1}}%
\newcommand{\hlkwd}[1]{\textcolor[rgb]{0.737,0.353,0.396}{\textbf{#1}}}%

\usepackage{framed}
\makeatletter
\newenvironment{kframe}{%
 \def\at@end@of@kframe{}%
 \ifinner\ifhmode%
  \def\at@end@of@kframe{\end{minipage}}%
  \begin{minipage}{\columnwidth}%
 \fi\fi%
 \def\FrameCommand##1{\hskip\@totalleftmargin \hskip-\fboxsep
 \colorbox{shadecolor}{##1}\hskip-\fboxsep
     % There is no \\@totalrightmargin, so:
     \hskip-\linewidth \hskip-\@totalleftmargin \hskip\columnwidth}%
 \MakeFramed {\advance\hsize-\width
   \@totalleftmargin\z@ \linewidth\hsize
   \@setminipage}}%
 {\par\unskip\endMakeFramed%
 \at@end@of@kframe}
\makeatother

\definecolor{shadecolor}{rgb}{.97, .97, .97}
\definecolor{messagecolor}{rgb}{0, 0, 0}
\definecolor{warningcolor}{rgb}{1, 0, 1}
\definecolor{errorcolor}{rgb}{1, 0, 0}
\newenvironment{knitrout}{}{} % an empty environment to be redefined in TeX

\usepackage{alltt}
\usepackage{bm}

\author{Christopher H. Jackson \\ MRC Biostatistics Unit, Cambridge, UK \\ \email{chris.jackson@mrc-bsu.cam.ac.uk}}
\title{flexsurv: flexible parametric survival modelling in R}

\Abstract{ \pkg{flexsurv} is an R package for fully-parametric modelling of 
  survival data.  Any parametric time-to-event distribution
  may be fitted if the user supplies at minimum a probability density
  or hazard function.  Many standard survival distributions are built
  in, and also the three and four-parameter generalized gamma and F
  models.  Any parameter of the distribution can be modelled as a
  linear or log-linear function of covariates.  Another built-in model
  is the spline model of Royston and Parmar, in which both baseline
  survival and covariate effects can be arbitrarily flexible
  parametric functions of time.
  
  \pkg{flexsurv} is intended to be similar to \pkg{survival}:
  right-censoring or left-truncation are specified in \code{Surv}
  objects, and the main model-fitting function, \code{flexsurvreg},
  uses the familiar syntax of \code{survreg}.  \pkg{flexsurv} also
  enhances the \pkg{mstate} package (Putter et al) by providing
  cumulative incidences for fully-parametric multi-state models. 
}
%\Keywords{survival}
\IfFileExists{upquote.sty}{\usepackage{upquote}}{}
\begin{document}

\section{Package motivation and design}

adv of parametric over cox. examples in HE
ref stata
based on survival.  right cens, left trunc,

The \code{survreg} function in \pkg{survival} only supports
two-parameter (location/scale) distributions, though users can supply
their own distributions.

Stata has a nice \code{streg}.  the \code{stpm2} spline model
More generally \code{stgenreg} has general.   ours can avoid num integ
The \code{flexsurv} is similar in spirit

review other packages in survival view: many model-specific packages

\section{Generic parametric survival models}

\subsection{Definitions} 

The general model that \pkg{flexsurv} fits has probability density function
\begin{equation}
  \label{eq:model}
f(t | \mu(\mathbf{z}), \bm{\alpha}(\mathbf{z})), \quad t \geq 0  
\end{equation}

$\mu=\alpha_0$ is the parameter of primary interest,
which usually governs the mean or location of the distribution.  Other
parameters $\mathbf{\alpha} = \alpha_1, \ldots, \alpha_R$ are called
``ancillary'' and determine the variance, shape or higher moments.
All parameters may depend on a vector of covariates $\mathbf{z}$
through link-transformed linear models $g_0(\mu) = \bm{\gamma}_0^{'}
\mathbf{z}$ and $g_r(\alpha_r) = \bm{\gamma}_r^{'} \mathbf{z}$. $g(a)$ will
typically be $\log(a)$ if $a$ is defined to be positive, or $g(a)=a$
if $a$ is unrestricted.

We also define (suppressing the conditioning for clarity) the
cumulative distribution function $F(t)$, survivor function $S(t) = 1 -
F(t)$, cumulative hazard $H(t) = -\log S(t)$ and hazard $h(t) =
f(t)/S(t)$.

Let $t_i: i=1,\ldots n$ be a sample of times from individuals $i$.
Let $c_i=1$ if $t_i$ is an observed death time, or $c_i=0$ if $t_i$ is
a right-censoring time, thus the true death time is known only to be
greater than $t_i$.  Also let $s_i$ be corresponding left-truncation
times, meaning that individual $i$ is only observed conditionally on
survival up to $s_i$, thus $s_i=0$ if there is no left-truncation.

The likelihood for the parameters in model (\ref{eq:model}), given the corresponding data vectors, is 
\[
l(\mu,\bm{\alpha} | \mathbf{t},\mathbf{c},\mathbf{s}) = \left\{ \prod_{i:\ c_i=1} f_i(t_i) \prod_{i:\ c_i=0} S_i(t_i) \right\} / \prod_i S_i(s_i)
\]

EXAMPLE DATASET HERE. 
bc?  not oral or franc.  ask unit, rita3? latimer?


\subsection{Other software} 

\subsection{Model fitting syntax} 

The main model-fitting function is called \code{flexsurvreg}.  Its
first argument is an R \emph{formula} object.  The left hand side of
the formula is defined by a \code{Surv} function from the
\pkg{survival} package.  This indicates here that the response
variable is \code{recyrs}, and that these are observed death and
censoring times when the variable \code{censrec} is 1 or 0
respectively.  If we also had left-truncation times in a variable
called \code{start}, the response would be
\code{Surv(start,recyrs,censrec)}.  All of these variables are in the
data frame called \code{bc}.

In order to fit a model, needs to know at least its probability density function. 
\begin{knitrout}
\definecolor{shadecolor}{rgb}{0.969, 0.969, 0.969}\color{fgcolor}\begin{kframe}
\begin{alltt}
\hlkwd{library}\hlstd{(flexsurv)}
\hlkwd{flexsurvreg}\hlstd{(}\hlkwd{Surv}\hlstd{(recyrs, censrec)} \hlopt{~} \hlstd{group,} \hlkwc{data}\hlstd{=bc,} \hlkwc{dist}\hlstd{=}\hlstr{"weibull"}\hlstd{)}
\end{alltt}


{\ttfamily\noindent\color{warningcolor}{\#\# Warning: NaNs produced\\\#\# Warning: NaNs produced\\\#\# Warning: NaNs produced}}\begin{verbatim}
## 
## Call:
## flexsurvreg(formula = Surv(recyrs, censrec) ~ group, data = bc,     dist = "weibull")
## 
## Estimates: 
##              data mean  est      L95%     U95%     se       exp(est)
## shape             NA     1.3797   1.2548   1.5170   0.0668       NA 
## scale             NA    11.4229   9.1818  14.2110   1.2728       NA 
## groupMedium   0.3338    -0.6136  -0.8623  -0.3649   0.1269   0.5414 
## groupPoor     0.3324    -1.2122  -1.4583  -0.9661   0.1256   0.2975 
##              L95%     U95%   
## shape             NA       NA
## scale             NA       NA
## groupMedium   0.4222   0.6943
## groupPoor     0.2326   0.3806
## 
## N = 686,  Events: 299,  Censored: 387
## Total time at risk: 2113
## Log-likelihood = -811.9, df = 4
## AIC = 1632
\end{verbatim}
\end{kframe}
\end{knitrout}


\subsection{Built-in survival distributions}
%% survreg also has normal, logistic, t for log time 

If the argument \code{dist} is a string, this denotes a built-in
survival distribution.  The built-in distributions are listed in Table
\ref{tab:dists}.  In each case the parameterisation of the
distribution and the probability density is defined by a function of
the same name as code{dist}, but preceded with \code{d}, for example
if \code{dist="weibull"}, the density function is
\code{dweibull}. For the exponential (\code{dexp}), Weibull, gamma
(\code{dgamma}) and log-normal (\code{dlnorm}), these are provided
with standard R installations.  For all other models these are
supplied in \pkg{flexsurv}


In addition \pkg{flexsurv} 
defines the generalized gamma and generalized F distributions, each in two parameterisations
\citet{ccox:taxonomy:hazards}
\citet{prentice:loggamma}
\citet{stacy:gengamma}
\citet{ccox:genf}
\citet{prentice:genf}
and a Gompertz distribution
useful as reduce
stable versions

For all built-in distributions, \pkg{flexsurv} also defines functions beginning h or H for the hazard or cumulative hazard. 

\begin{table}
  \begin{tabular}{lll}
\hline
    &  Parameters &  Density function \\
\hline
    Exponential & rate             & dexp   \\
    Weibull     & shape, scale     & dweibull \\
    Gamma       & shape, rate      & dgamma\\
    Log-normal  & meanlog, sdlog   & dlnorm\\
    Gompertz    & shape, rate      & dgompertz \\
    Generalized gamma (Prentice)   & & dgengamma \\
    Generalized gamma (Stacy 1962) & & dgengamma.orig \\
    Generalized F     (Prentice)   & & dgenf \\
    Generalized F                  & & dgenf.orig \\
\hline
  \end{tabular}
  \caption{Built-in parametric survival distributions in \pkg{flexsurv}}
  \label{tab:dists}
\end{table}


\subsection{Supplying own distributions}

Suppose we know the distribution 

Many contributed R packages contain density and cumulative
distribution functions for positive distributions (ref eha, VGAM,
ActuDistns)

User-specified p function, d function, hazard
since loads of contributed distributions
fixed-dimension.   In Section \ref{sec:gdim} 

Distribution exists in another package, but may be parameterised 

Example: Gompertz-Makeham


optimisation methods, derivatives , parallel processing with pnmath

Demo on at least one dataset: stgenreg uses bc example i think

test if we can do the gen gamma prop haz trick in stgenreg paper

Basic Weibull prop haz model in stgenreg.  Advantage is that it's just as fast as R built in stuff.



\subsection{Output functions}

\code{summary.flexsurvreg} calculates the estimated survival, hazard
or cumulative hazard at a series of times and for specified covariate
values. Confidence intervals are produced by simulating a large sample
from the asymptotic normal distribution of the maximum likelihood
estimates $\gamma$ OR WHATEVER, via the function
\code{normboot.flexsurvreg}.  The default \code{plot} method for
\code{flexsurvreg} objects graphs these fitted trajectories against
non-parametric estimates based on Kaplan-Meier or kernel estimation
(REF muhaz), while the \code{lines} method adds lines to an existing
plot.  REFER TO EXAMPLE FIGURE

Any user-defined function of the basic model parameters $\gamma$ OR
WHATEVER and time can also be summarised in the same way.  For
example, in a non-proportional hazards model, the hazard ratio between
two groups of interest varies through time.  To plot this trajectory,
and confidence intervals.   EXAMPLE FROM SPLINE. 

Restricted mean survival: say of interest. ref royston + parmar


\section{Spline models}

parameters are vectors, different design

relation to fractional polynomials 
(see \pkg{mfp} for continuous covariates, slightly diff)

stgenreg has demo of spline modelling on the log hazard scale.  Can we do this using a generic distribution? 
(advantage: when there are multiple time dependent effects, the
interpretation of the time-dependent hazard ratios is simplified as
they do not depend on values of other covariates, which is the case
when modelling on the cumulative hazard scale (Royston and Lambert
2011).

Demo on at least one dataset 

\subsection{General-dimension models}
\label{sec:gdim}

The spline model above is an example of a model where the general
parametric form can be written explicitly as in model \ref{eq:model},
but the length of alpha is arbitrary.  semi-parametric 

\pkg{flexsurv} has the tools to deal with any 

where are vectors



\section{Multi-state models}

enhances mstate

% cif in comp risks planned for stgenreg 
\section{Potential extensions}

relative survival
interval censoring
frailty 
what else does survival do 


\appendix
\section{Acknowledgements}
Thanks to Milan Bouchet-Valat.

\bibliography{survival}

\end{document}
